\input{mmd-beamer-header}
\def\mytitle{What is MultiMarkdown?}
\def\subtitle{And why should you care?}
\def\myauthor{Fletcher T. Penney}
\def\affiliation{http:\slash \slash fletcherpenney.net\slash multimarkdown\slash }
\def\copyright{2009 Fletcher T. Penney.  \\
This work is licensed under a Creative Commons License.  \\
http:\slash \slash creativecommons.org\slash licenses\slash by-sa\slash 2.5\slash }
\def\latexmode{beamer}
\def\bibliostyle{fletcher}
\def\theme{keynote-gradient}
\input{mmd-beamer-begin-doc}
\begin{frame}

\frametitle{MultiMarkdown is a derivative of Markdown}
\label{multimarkdownisaderivativeofmarkdown}

\href{http://daringfireball.net/projects/markdown/}{Markdown}\footnote{\href{http://daringfireball.net/projects/markdown/}{http:\slash \slash daringfireball.net\slash projects\slash markdown\slash }} is a program and a
syntax by John Gruber that allows you to easily convert plain text into HTML
suitable for using on a web page.

\end{frame}

\begin{frame}[fragile]

\frametitle{The old way was complicated}
\label{theoldwaywascomplicated}

\begin{verbatim}

<p>In order to create valid 
<a href="http://en.wikipedia.org/wiki/HTML">HTML</a>, you 
need properly coded syntax that can be cumbersome for 
&#8220;non-programmers&#8221; to write. Sometimes, you
just want to easily make certain words <strong>bold
</strong>, and certain words <em>italicized</em> without
having to remember the syntax. Additionally, for example,
creating lists:</p>

<ul>
<li>should be easy</li>
<li>should not involve programming</li>
</ul>

\end{verbatim}


\end{frame}

\begin{frame}[fragile]

\frametitle{The new way is easier}
\label{thenewwayiseasier}

\begin{verbatim}

In order to create valid [HTML][], you need properly
coded syntax that can be cumbersome for 
"non-programmers" to write. Sometimes, you just want
to easily make certain words **bold**, and certain 
words *italicized* without having to remember the 
syntax. Additionally, for example, creating lists:

* should be easy
* should not involve programming

[HTML]: http://en.wikipedia.org/wiki/HTML

\end{verbatim}


\end{frame}

\begin{frame}

\frametitle{Markdown is designed for people}
\label{markdownisdesignedforpeople}

The overriding design goal for Markdown's formatting syntax is to make it as
readable as possible. The idea is that a Markdown-formatted document should be
publishable as-is, as plain text, without looking like it's been marked up
with tags or formatting instructions. While Markdown's syntax has been
influenced by several existing text-to-HTML filters, the single biggest source
of inspiration for Markdown's syntax is the format of plain text email.
~\cite{Gruber}

\end{frame}

\begin{frame}

\frametitle{But Markdown wasn't complete}
\label{butmarkdownwasntcomplete}

I, and others, loved the spirit and elegance of Markdown, but felt it was
still missing a few features that each of us considered were essential.
Several variations on Markdown arose to meet the needs of these other
programmers.

\end{frame}

\begin{frame}

\frametitle{MultiMarkdown adds several new features}
\label{multimarkdownaddsseveralnewfeatures}

\begin{itemize}
\item footnotes

\item tables

\item citations and bibliography

\item image attributes

\item metadata

\item internal cross-references

\item ASCIIMathML support

\item glossary entries

\item definition lists

\item and more{\ldots}

\end{itemize}

\end{frame}

\begin{frame}

\frametitle{MMD also adds something else{\ldots}}
\label{mmdalsoaddssomethingelse}

\begin{itemize}
\item Outside of the actual syntax, MMD supports the use of
 \href{http://en.wikipedia.org/wiki/XSL_Transformations}{XSLT}\footnote{\href{http://en.wikipedia.org/wiki/XSL_Transformations}{http:\slash \slash en.wikipedia.org\slash wiki\slash XSL\_Transformations}} to convert the
 XHTML output into something else, e.g.
 \href{http://en.wikipedia.org/wiki/LaTeX}{LaTeX}\footnote{\href{http://en.wikipedia.org/wiki/LaTeX}{http:\slash \slash en.wikipedia.org\slash wiki\slash LaTeX}}.

\item This allows you to use the same markup language (MultiMarkdown) to create a
 high quality pdf (article, book, or presentation like this one) without any
 additional programming.

\item Most importantly, you don't have to know how this works, or even what it
 the LaTeX commands mean --- just have the software installed.

\end{itemize}

\end{frame}

\begin{frame}

\frametitle{So, one text file becomes multiple final documents}
\label{soonetextfilebecomesmultiplefinaldocuments}

\begin{figure}[htbp]
\centering
\includegraphics[keepaspectratio,width=\textwidth,height=0.75\textheight]{MMDTree.pdf}
\label{}
\end{figure}


\end{frame}

\begin{frame}

\frametitle{The goal is to separate content from formatting}
\label{thegoalistoseparatecontentfromformatting}

By focusing on the text content of your document, you can focus on the
creative, the scientific, the \emph{human}. Let your computer do what it is good at
--- the fairly boring job of making sure that margins are correct, that
paragraphs are properly separated, your footnotes are in order, and that your
tables line up --- regardless of the final format you want your document to
take.

\end{frame}

\begin{frame}[fragile]

\frametitle{ASCIIMathML --- it's like Markdown for mathematics}
\label{asciimathml---itslikemarkdownformathematics}

Built into MultiMarkdown is support for
\href{http://asciimathml.sourceforge.net/}{ASCIIMathML}\footnote{\href{http://asciimathml.sourceforge.net/}{http:\slash \slash asciimathml.sourceforge.net\slash }} --- a tool that allows you
to write mathematical equations in plain text, yet produce high quality
output. It can be used for web pages (if your browser supports MathML), or for
LaTeX.

\begin{verbatim}

\\[ {e}^{i\pi }+1=0 \\]

\end{verbatim}


becomes

\[ {e}^{i\pi }+1=0 \]

\end{frame}

\begin{frame}[fragile]

\frametitle{Images are just as easy}
\label{imagesarejustaseasy}

\begin{verbatim}

![Nautilus Star](Nautilus_Star.png)

\end{verbatim}


becomes{\ldots}

\end{frame}

\begin{frame}

\frametitle{Images are just as easy}
\label{imagesarejustaseasy}

\begin{figure}[htbp]
\centering
\includegraphics[keepaspectratio,width=\textwidth,height=0.75\textheight]{Nautilus_Star.png}
\label{}
\end{figure}


\end{frame}

\begin{frame}

\frametitle{Support for a bibliography is also included}
\label{supportforabibliographyisalsoincluded}

\begin{itemize}
\item MultiMarkdown has support for \href{http://www.bibtex.org/}{BibTeX}\footnote{\href{http://www.bibtex.org/}{http:\slash \slash www.bibtex.org\slash }}, or
 for just including your own citations, so that you can back up your
 arguments.~\cite[p. 42]{fake}

\item The citation above links to the corresponding entry in the bibliography.

\end{itemize}

\end{frame}

\begin{frame}

\frametitle{Installation is easy}
\label{installationiseasy}

\begin{itemize}
\item Download the MultiMarkdown software.

\item If you want to use LaTeX, install a version appropriate for your operating
 system.

\item If you're running Windows you will need to install Perl. If you want to use
 LaTeX or the more advanced XHTML XSLT files, you need to install xsltproc.
 These are included in Mac OS X, and default installs of most Linux
 distributions.

\end{itemize}

More complete instructions are available at the \href{http://fletcherpenney.net/mmd/users_guide/quickstart_guide_to_multimarkd/}{MultiMarkdown site}\footnote{\href{http://fletcherpenney.net/mmd/users_guide/quickstart_guide_to_multimarkd/}{http:\slash \slash fletcherpenney.net\slash mmd\slash users\_guide\slash quickstart\_guide\_to\_multimarkd\slash }}.

\end{frame}

\begin{frame}

\frametitle{How do I create a MultiMarkdown document?}
\label{howdoicreateamultimarkdowndocument}

\begin{itemize}
\item A MultiMarkdown is simply a text document that is written in the
 MultiMarkdown syntax. You can use any text editor or application you like.
 If your editor supports fonts, italics, etc. then be sure to save as a plain
 text file (not a .doc, RTF, or other ``rich'' format).

\item Some applications include built-in support for MultiMarkdown in various
 ways. There's a \href{http://fletcher.github.com/markdown.tmbundle/}{bundle}\footnote{\href{http://fletcher.github.com/markdown.tmbundle/}{http:\slash \slash fletcher.github.com\slash markdown.tmbundle\slash }} for \href{http://macromates.com/}{TextMate}\footnote{\href{http://macromates.com/}{http:\slash \slash macromates.com\slash }}, and \href{http://www.literatureandlatte.com/scrivener.html}{Scrivener}\footnote{\href{http://www.literatureandlatte.com/scrivener.html}{http:\slash \slash www.literatureandlatte.com\slash scrivener.html}} includes
 MultiMarkdown support.

\end{itemize}

\end{frame}

\begin{frame}

\frametitle{Why should I mess with this LaTeX stuff?}
\label{whyshouldimesswiththislatexstuff}

MultiMarkdown's support for LaTeX is designed to automatically do the ``right''
thing in most situations for most people. But if you want to dig in and learn
more, you can customize MMD to create highly tailored documents that suit your
specific needs. If you want high quality typography, LaTeX is the way to go.

If you already know what LaTeX is, then MultiMarkdown allows you to create
documents without learning all of the complicated LaTeX commands and markup.

\end{frame}

\begin{frame}

\frametitle{How do I create a fancy PDF?}
\label{howdoicreateafancypdf}

If you're using LaTeX, and have the proper software installed it's easy:

\begin{enumerate}
\item Type \texttt{mmd2pdf filename.txt}

\item There is no step 2

\end{enumerate}

\end{frame}

\begin{frame}

\frametitle{Where to learn more}
\label{wheretolearnmore}

\begin{itemize}
\item \href{http://fletcherpenney.net/multimarkdown/}{http:\slash \slash fletcherpenney.net\slash multimarkdown\slash }

\item \href{http://groups.google.com/group/multimarkdown/}{http:\slash \slash groups.google.com\slash group\slash multimarkdown\slash }

\item \href{http://fletcher.github.com/MultiMarkdown-Gallery/}{http:\slash \slash fletcher.github.com\slash MultiMarkdown-Gallery\slash }

\end{itemize}

\end{frame}

\begin{frame}

\frametitle{By the way{\ldots}}
\label{bytheway}

This presentation was, of course, written in MultiMarkdown and processed by
typing \texttt{mmd2pdf what\_is\_mmd.txt}.

It uses the \texttt{beamer} XSLT file, and the \texttt{keynote-gradient} beamer theme.

\end{frame}

\part{Bibliography}
\begin{frame}[allowframebreaks]
\frametitle{Bibliography}
\def\newblock{}
\begin{thebibliography}{0}
\bibitem{Gruber}
John Gruber. Daring Fireball: Markdown. [Cited January 2006]. Available from \href{http://daringfireball.net/projects/markdown/}{http:\slash \slash daringfireball.net\slash projects\slash markdown\slash }.


\bibitem{fake}
John Doe. \emph{A Totally Fake Book}. Vanity Press, 2006.


\end{thebibliography}
\end{frame}

\mode<all>
\input{mmd-beamer-footer}

\end{document}\mode*

